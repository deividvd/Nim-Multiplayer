\section{Requisiti}

\subsection{Funzionali}
\label{requisiti funzionali}

\begin{itemize}
\item
	Registrazione di un nuovo utente e servizio di login.
\item
	Creazione di una stanza di gioco e relativi inviti.
\item
	Configurare le opzioni di gioco, durante la creazione di una nuova partita.
	\begin{itemize}
	\item
		Scegliere il numero di giocatori.
	\item
		Scegliere il numero di righe di bastoncini.
	\item
		Scegliere la rotazione dei turni: rotazione oppure chaos.
	\item
		Scegliere la modalità di vittoria: Standard oppure Marienbad.
	\end{itemize}
\item
	Gestione di una partita Nim tradizionale (1 vs 1): svolgimento del turno; rotazione dei turni (2 tipologie); modalità di vittoria (2 tipologie).
	\begin{itemize}
	\item
		Svolgimento del turno: durante il proprio turno, la mossa consiste nel rimuovere un numero qualsiasi di bastoncini posti su una qualsiasi e unica riga, ma tali bastoncini devono essere adiacenti e tra loro non vi devono essere altri bastoncini già rimossi (non è possibile saltare la mossa nel proprio turno).
	\item
		Rotazione dei turni (2 tipologie).
		\begin{itemize}
		\item
			La tipica rotazione: i giocatori si alternano in un ordine prestabilito per tutta la durata della partita, generato casualmente ad inizio partita.
		\item
			Turni \emph{chaos}: il gioco procede a turni casuali, ovvero il gioco sceglierà di far passare il turno casualmente ad un giocatore fra quelli che sono "indietro di un turno".
		\end{itemize}
	\item
		Modalità di vittoria (2 tipologie).
		\begin{itemize}
		\item
			Standard: il giocatore che rimuove l'ultimo bastoncino vince.
		\item
			Marienbad: il giocatore che rimuove l'ultimo bastoncino perde.
		\end{itemize}
	\end{itemize}
\item
	Gestione di una partita multiplayer (fino a 6 giocatori).
	\begin{itemize}
	\item
		Svolgimento del turno e rotazione dei turni: sono identici ai concetti descritti nei punti precedenti.
	\item
		Modalità di vittoria (2 tipologie).
		\begin{itemize}
		\item
			Standard: identica a 1 vs 1 (il giocatore che rimuove l'ultimo bastoncino vince, e la partita finisce).
		\item
			Marienbad: il giocatore che rimuove l'ultimo bastoncino viene eliminato dalla partita, ed il gioco prosegue nel seguente modo finché non rimarrà un solo giocatore (vincitore). Dopo l'eliminazione di un giocatore, i bastoncini nelle righe vengono tutti ripristinati, ad eccezione dei bastoncini rimossi dai giocatori eliminati.
		\end{itemize}
	
	\newpage
	
	\item
		Gestione delle disconnessioni dei giocatori. Le disconnessioni vengono gestite quando si presenta il turno del giocatore disconnesso.
		\begin{itemize}
		\item
			Vittoria Standard: quando si verifica la disconnessione di un giocatore, quest'ultimo viene eliminato dalla partita senza alterare i bastoncini in gioco.
		\item
			Vittoria Marienbad: la disconnessione di un giocatore è considerata come una eliminazione, per cui verranno ripristinati i bastoncini opportuni.
		\end{itemize}
	\end{itemize}
\end{itemize}

\subsection{Non Funzionali}

\begin{itemize}
\item
	Connessione sicura tramite https (implementato).
\item
	Memorizzazione cifrata delle password nel database (implementato).
\item
	Recupero delle credenziali utente, tramite invio di una mail (non implementato).
\item
	Giocare una partita come guest non registrato (non implementato).
\item
	Osservare una partita (non implementato).
\item
	Aggiungere nel server un task dedicato per identificare periodicamente le partite finite e le partite mai iniziate, per poi cancellarle e ridurre ragionevolmente lo spazio di archiviazione del database (non implementato).
\end{itemize}

\subsection{Requisiti Tecnologici}

L'applicazione verrà realizzata con il solution stack MEVN.

Le comunicazioni real time riguardo la creazione e la gestione della stanza di gioco verranno implementate con la libreria Socket.IO.

\subsection{Requisiti degli Utenti}

Il metodo di design scelto è \emph{user centered design} (UCD), il quale pone il focus sui bisogni degli utenti in modo iterativo, al fine di massimizzare il grado di soddisfazione della \emph{user experience} (UX).

Per questo è stata svolta un'analisi dei \emph{target user}, e sono state simulate alcune \emph{Personas} come modelli "virtuali" di utenti, correlate ad uno \emph{scenario d'uso} specifico. Ecco la lista di \emph{Personas} considerate.

\begin{itemize}
\item
	Simone conosce la versione tradizionale di Nim e conosce il "segreto" celato dietro la "complessità" del gioco.
	\newline
	Simone gioca partite 1 vs 1, con modalità di vittoria standard, e turni svolti rigorosamente in rotazione.
\item
	Eraldo ha giocato sempre e solo la versione tradizionale di Nim con modalità di vittoria Marienbad. Ripudia la vittoria standard perché secondo lui non è la versione originale, pertanto non deve essere giocata.
	\newline
	Eraldo gioca partite 1 vs 1, con modalità di vittoria Marienbad, e turni svolti in rotazione.
\item
	Ilaria è completamente estranea a giochi matematici e combinatori, tuttavia vorrebbe comunque essere partecipe e giocare insieme ai suoi amici. Talvolta vorrebbe anche poter vincere senza necessariamente dover capire la classe di complessità del gioco.
	\newline
	Ilaria gioca partite multiplayer con gli amici, con qualsiasi modalità di vittoria, e turni svolti in modalità chaos.
\item
	Lorenzo è il classico nerd che conosce a fondo i giochi matematici e giochi strategici che includono il calcolo combinatorio, a volte si cimenta in tornei di scacchi piuttosto impegnativi. Per lui il Nim è un gioco estremamente semplice e noioso.
	\newline
	Lorenzo gioca partite multiplayer con il numero massimo di giocatori consentito, con modalità di vittoria Marienbad (perché le partite sono più lunghe e richiedono un piccolo sforzo mnemonico), e turni svolti rigorosamente in rotazione.
\end{itemize}

\newpage
